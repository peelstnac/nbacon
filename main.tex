\documentclass{article}
\usepackage{wrapfig}
\usepackage{setspace}
\usepackage{calc}
\usepackage{multicol}
\usepackage{cancel}
\usepackage[retainorgcmds]{IEEEtrantools}
\usepackage[margin=3cm]{geometry}
\usepackage{amsmath}
\newlength{\tabcont}
\setlength{\parindent}{0.0in}
\setlength{\parskip}{0.05in}
\usepackage{empheq}
\usepackage{framed}
\usepackage[most]{tcolorbox}
\usepackage{xcolor}
\usepackage{amsmath,amsthm,amssymb,scrextend}
\usepackage{fancyhdr}
\colorlet{shadecolor}{orange!15}
\parindent 0in
\parskip 12pt
\geometry{margin=1in, headsep=0.25in}
\newtheorem{defn}{Definition}
\newtheorem{exer}{Exercise}
\newtheorem{theorem}{Theorem}
\newtheorem{lemma}{Lemma}
\renewcommand{\qed}{\hfill$\blacksquare$}
\let\newproof\proof
\renewenvironment{proof}{\begin{addmargin}[1em]{0em}\begin{newproof}}{\end{newproof}\end{addmargin}\qed}
\begin{document}
\title{}
\author{}
\maketitle

\section{Power Series} 
\begin{defn} 
Power series are of the form $$\sum_{n=0}^\infty a_nx^n = a_0+a_1x+a_2x^2+a_3x^3+...$$ 
\end{defn}

The goal of this section is to explain the circumstances where the continuity and differentiability properties of polynomials are kept with a power series (think polynomial of infinite degree). 

\begin{shaded} 
\begin{enumerate} 
    \item Consider sequence of functions $g_n(x)=x^n$ on $[0, 1].$ Now $$\lim_{n \rightarrow \infty} g_n(x) = g(x)$$ where $g(x)=0$ if $0 \le x < 1$ and $g(x)=1$ if $x=1.$ Here, we see example of sequence of continuous functions converging to $g(x)$ which is discontinuous at $g(1).$
    \item Another example of the limit function not inheriting its sequence's properties can be seen with $h_n(x) = x^{\frac{2n}{2n-1}}.$ Since $$h(x) = \lim_{n \rightarrow \infty} h_n(x) = |x|,$$ we see that $h$ is not differentiable at $h(0).$
\end{enumerate}
\end{shaded}

\begin{defn}
A sequence of functions $f_n: X \rightarrow R$ converges pointwise to $f$ on $X$ if for all $x \in X,$ the sequence $f_n(x)$ converges to $f(x).$
\end{defn}

\begin{defn}
Our sequence $f_n$ converges uniformly to $f$ on $X$ if for any $\epsilon > 0,$ there exists $N$ such that $|f_{n \ge N}(x)-f(x)|<\epsilon$ for all $x \in X.$
\end{defn}
\noindent Since with pointwise convergence, neither continuity nor differentiability are guaranteed to be preserved, we consider uniform convergence.

\begin{theorem}{(Cauchy Criterion for Uniform Convergence)}
If for any $\epsilon > 0,$ there exist $N$ such that $|f_{n \ge N}(x) - f_{m \ge N}(x)| < \epsilon,$ then $(f_n) \rightarrow f$ uniformly.
\end{theorem}
\begin{proof}
Suppose the above were true. Fix an arbitrary $x \in X.$ For any $\epsilon > 0,$ there is $N$ where $|f_{n \ge N}(x) - f_{m \ge N}(x)| < \epsilon.$ By Cauchy Criterion for Convergent Sequences, $(f_n(x))$ converges to some value $f(x).$ Since $N$ is not dependant on $x,$ we can say this is true for all $x \in X$ and so $(f_n(x)) \rightarrow f(x)$ uniformly.
\end{proof}
\end{document}
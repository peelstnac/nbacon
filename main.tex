\documentclass{article}
\usepackage{wrapfig}
\usepackage{setspace}
\usepackage{calc}
\usepackage{multicol}
\usepackage{cancel}
\usepackage[retainorgcmds]{IEEEtrantools}
\usepackage[margin=3cm]{geometry}
\usepackage{amsmath}
\newlength{\tabcont}
\setlength{\parindent}{0.0in}
\setlength{\parskip}{0.05in}
\usepackage{empheq}
\usepackage{framed}
\usepackage[most]{tcolorbox}
\usepackage{xcolor}
\usepackage{amsmath,amsthm,amssymb,scrextend}
\usepackage{fancyhdr}
\colorlet{shadecolor}{orange!15}
\parindent 0in
\parskip 12pt
\geometry{margin=1in, headsep=0.25in}
\newtheorem{defn}{Definition}
\newtheorem{exer}{Exercise}
\newtheorem{theorem}{Theorem}
\newtheorem{lemma}{Lemma}
\renewcommand{\qed}{\hfill$\blacksquare$}
\let\newproof\proof
\renewenvironment{proof}{\begin{addmargin}[1em]{0em}\begin{newproof}}{\end{newproof}\end{addmargin}\qed}
\begin{document}
\title{}
\author{}
\maketitle

\section{Power Series} 
\begin{defn} 
Power series are of the form $$\sum_{n=0}^\infty a_nx^n = a_0+a_1x+a_2x^2+a_3x^3+...$$ 
\end{defn}

The goal of this section is to explain the circumstances where the continuity and differentiability properties of polynomials are kept with a power series (think polynomial of infinite degree). 

\begin{shaded} 
\begin{enumerate} 
    \item Consider sequence of functions $g_n(x)=x^n$ on $[0, 1].$ Now $$\lim_{n \rightarrow \infty} g_n(x) = g(x)$$ where $g(x)=0$ if $0 \le x < 1$ and $g(x)=1$ if $x=1.$ Here, we see example of sequence of continuous functions converging to $g(x)$ which is discontinuous at $g(1).$
    \item Another example of the limit function not inheriting its sequence's properties can be seen with $h_n(x) = x^{\frac{2n}{2n-1}}.$ Since $$h(x) = \lim_{n \rightarrow \infty} h_n(x) = |x|,$$ we see that $h$ is not differentiable at $h(0).$
\end{enumerate}
\end{shaded}

\begin{defn}
A sequence of functions $f_n: X \rightarrow R$ converges pointwise to $f$ on $X$ if for all $x \in X,$ the sequence $f_n(x)$ converges to $f(x).$
\end{defn}

\begin{defn}
Our sequence $f_n$ converges uniformly to $f$ on $X$ if for any $\epsilon > 0,$ there exists $N$ such that $|f_{n \ge N}(x)-f(x)|<\epsilon$ for all $x \in X.$
\end{defn}
\noindent Since with pointwise convergence, neither continuity nor differentiability are guaranteed to be preserved, we consider uniform convergence.

\begin{theorem}{(Cauchy Criterion for Uniform Convergence)}
If for any $\epsilon > 0,$ there exist $N$ such that $|f_{n \ge N}(x) - f_{m \ge N}(x)| < \epsilon,$ then $(f_n) \rightarrow f$ uniformly.
\end{theorem}
\begin{proof}
Suppose the above were true. Fix an arbitrary $x \in X.$ For any $\epsilon > 0,$ there is $N$ where $|f_{n \ge N}(x) - f_{m \ge N}(x)| < \epsilon.$ By Cauchy Criterion for Convergent Sequences, $(f_n(x))$ converges to some value $f(x).$ Since $N$ is not dependant on $x,$ we can say this is true for all $x \in X$ and so $(f_n(x)) \rightarrow f(x)$ uniformly.
\end{proof}

\noindent Indeed, continuity is preserved with uniform convergence.
\begin{theorem}
Suppose each function in the sequence of functions $f_n:X \rightarrow R$ is continuous at $c \in X.$ If $(f_n(x)) \rightarrow f(x)$ uniformly, then $f$ is continuous at $c.$
\end{theorem}
\begin{proof}
Fix $\frac{\epsilon}{3} > 0.$ There must exist $N$ such that $|f_N(x)-f(x)|<\frac{\epsilon}{3}$ and $|f_N(c)-f(c)|<\frac{\epsilon}{3}.$ Furthermore, since $f_N$ is continuous, it follows that there exists $\delta$ where $|x-c|<\delta \implies |f_N(x)-f_N(c)|<\frac{\epsilon}{3}.$ It follows that $$|f(x)-f(c)+f_N(x)-f_N(x)+f_N(c)-f_N(c)| \le |f_N(x)-f(x)|+|f_N(c)-f(c)|+|f_N(x)-f_N(c)| < \epsilon.$$
\end{proof}

\begin{theorem}
Let $X = \{x_1, x_2, x_3, ...\},$ a countable set of real numbers. If there is a bounded sequence of functions $f_n:X \rightarrow R,$ it follows that there exist a pointwise convergent subsequence.
\end{theorem}
\begin{proof}
Consider the sequence $f_n(x_1).$ Since this sequence is bouded, by Bolzano-Weierstrass theorem, there is a convergent subsequence $F_n(x_1).$ Now consider $F_n(x_2)$ (ie. the functions part of the convergent subsequence for $x_1$ applied to $x_2$). Like with $x_1,$ we can generate a subsequence of $F_n$ (so we're nesting subsequences). Let us keep generating subsequences in this manner. Now, let us create a new sequence of functions in the following manner. Enumerate through $x_1, x_2, x_3, ...:$ for $x_1,$ pick $f_{k_1}$ where $f_{k_1}(x_1)$ is part of $x_1$'s subsequence. For $x_n,$ pick $f_{k_n}$ in the same manner but make sure that $k_n > k_{n-1}.$ Notice how for any point $x_i,$ the sequence $f_{k_i}(x_i), f_{k_{i+1}}(x_i), f_{k_{i+2}}(x_i)...$ is convergent. Therefore, $(f_{k_n})$ converges pointwise to some function.
\end{proof}

\noindent This is one condition where the idea of Bolzano-Weierstrass can be applied to sequences of functions. Do pay special attention to the fact that $f_n$ was bounded. We present another condition, but this time it implies uniform continuity.

\begin{defn}
A function $f_n:X \rightarrow R$ is called equicontinuous if for all $\epsilon > 0,$ there exist $\delta > 0$ such that $$|x-c| < \delta \implies |f_{n \ge N}(x) - f_{n \ge N}(c)| < \epsilon$$ for any $x$ and $c$ in domain.
\end{defn}
\noindent Note that this is stronger than uniform continuity since $\delta$ is decided for all the functions as well.
\begin{theorem}{(Arzela-Ascoli Theorem)}
Consider bounded sequence of functions $f_n:[A, B] \rightarrow R$ where each function is equicontinuous. There exists a uniformly convergent subsequence of $f_n.$
\end{theorem}
\begin{proof}
Fix $E > 0.$ Let $\epsilon = \frac{E}{2}.$ Consider the countable set of rational numbers within the domain $\{r_1, r_2, r_3, ...\}.$ From theorem 3, it follows that there exist subsequence of functions $f_{n_k}$ that is pointwise convergent on these rational numbers. Due to equicontinuity, there exist $\delta$ such that $$|x-c| < \delta \implies |f_n(x)-f_n(c)| < \epsilon$$ for all $x$ and $c$ in $[A, B].$ Now let $\{V_{\delta}(r_{n_1}), V_{\delta}(r_{n_2}), V_{\delta}(r_{n_3}), ..., V_{\delta}(r_{n_L})\}$ be a finite open cover of $[A, B]$ consisting of $\delta$ neighborhoods of $r_{n_i},$ who are rational numbers within $[A, B].$ For each $r_{n_i},$ let there be corresponding $N_i$ such that $|f_{n \ge N_i}(r_{n_i})-f_{m \ge N_i}(r_{n_i})| < \frac{\epsilon}{3}$ (we know such value exist from our proof about pointwise convergence earlier). Now, define $$N = \max \{N_1, N_2, N_3, ..., N_L\}.$$ Consider two real numbers $a, b$ in $[A, B].$ Because rational numbers are dense in the reals, we can pick two distinct rational numbers $R_1, R_2$ in the $\delta$ neighborhoods of $a$ and $b$ respectively. We know, from triangle inequality, that for $E > 0$ there exist $N$ such that
\begin{align*}
|f_{n \ge N}(a)-f_{m \ge N}(b)| &= |f_{n \ge N}(a)-f_{m \ge N}(b) - f_{n \ge N}(R_1) + f_{n \ge N}(R_1) - f_{m \ge N}(R_2) + f_{m \ge N}(R_2)| \\
&\le |f_{n \ge N}(a) - f_{n \ge N}(R_1)| + |f_{m \ge N}(b) - f_{n \ge N}(R_2)| + |f_{n \ge N}(R_1) - f_{m \ge N}(R_2)| \\
&\le \frac{\epsilon}{3} + \frac{\epsilon}{3} + \frac{\epsilon}{3} \\
&\le \epsilon \\
&< E.
\end{align*}
\end{proof}

\noindent Now we examine the conditions where differentiability is preserved.
\begin{theorem}
If $f_n:[A, B]$ is a sequence of differentiable functions with the sequence $(f'_n) \rightarrow g$ uniformly and for some $x \in [A, B]$ the sequence $f_n(x)$ is convergent, then $(f_n) \rightarrow f$ uniformly and $f'(x) = g(x).$
\end{theorem}
\begin{proof}

\end{proof}
\end{document}